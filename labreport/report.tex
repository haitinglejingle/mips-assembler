% Lab 7

\documentclass[titlepage]{article}
\usepackage[margin=1in]{geometry}
\usepackage{graphicx}
\usepackage{float}
\usepackage{color}

\title{EE126 Lab7, Fall17}
\author{Michelle Chan and Ashton Stephens, Tufts University}


\begin{document}

\begin{titlepage}
\maketitle
\end{titlepage}

\section*{Introduction}



% TODO: provide background here
% why a mips assembler? what's an assembler?

Although programming languages consist of syntax and semantics reminiscent 
of natural languages, all data in computers are encoded in binary bits that
hold the state of either `1' or `0' with arbitrary meaning attached. A process
of translating languages into machine code is needed and often performed by
other kinds of programs. 

Often, a high-level language may be compiled into an assembly language, the 
lowest-level of abstraction from binary. From assembly, an assembler translates 
assembly code into machine code in binary format readable to computers. 

MIPS is an instruction set architecture (ISA) introduced in 1985 often used in 
embedded systems and university courses. As a reduced instruction set 
arcitecture (RISC), most assembly instructions in this ISA are simple such that
the cycles per instruction are kept low. 

In this course, a 5-stage pipeline was built in VHDL. This MIPS 32-bit assembler
was built to help translate MIPS assembly instructions into machine code that 
can be easily loaded into instruction memory in VHDL. 

\section*{Specifications}
This MIPS-32 assembler reads from a file specified as a command line argument
or from stdin if a file is absent. The output is printed on stdout and can be
piped into a file.

This assembler recognizes all 32 MIPS-32 registers and labels. A list of 
supported instructions, including pseudo-instructions, are clearly listed in the

\section*{Implementation}

On the surface, the assembler conducts two passes where the first pass handles 
labels and the second pass translates each assembly line of code. 

\subsection*{Tables}
There were three tables used to lookup information used in the implementation.

1. The mneumonic table maps a mneumonic to a corresponding opcode.

2. The register table maps the register name to its number in memory.

3. The pseudoinstruction table maps a pseudoinstruction to the instructions
it expands into along with the order to pass in registers.

\subsection*{First Pass}
Each line of assembly code in the program is read line-by-line and stored 
internally to be read again from the second pass. At each line, the first word 
is checked for a label and the word address is incremented the appropriate 
amount. Labels are stored as the key to its word address value. The word address
that is tracked always increments by one on a MIPS instruction because there 
exists the invariant that each instruction is one word long. The first word is
also checked against the internal pseudoinstruction map to increment more than 
one word forward.

\subsection*{Second Pass}
In the second pass, the mneumonic and operands of each line are parsed into 
strings. Using the current word address and parsed strings, the assemble method 
is called to translate into machine code. 

Internally, the translation looks up the mneumonic to get either an Instruction
type or Pseudoinstruction type internally defined as structs containing all the
necessary information for assembling. In ``struct Mnemonic\_func,'' the 
information stored includes the opcode (or funct field), the instruction format 
(R-type, I-type, or J-type), and the syntax group. Each syntax group referred to
a unique order in which the operands for an assembly instruction were translated
and placed into the machine instruction. 

In the case that the mneumonic maps to a pseudoinstruction, the list of 
instructions that it expands to is returned along with the appropriate order of
operands for each instruction. Using this, the instructions are referenced in 
the map again.

After the translation, the binary is printed to stdout and the number of words 
to increment the program counter is also added into the word address. 

\section*{Results}

The assembler was tested using the test assembly programs from previous labs for
testing the 5-stage MIPS processor. The output in binary was decoded into MIPS 
assembly again and compared with the original input to check for the 
functionality of this assembler.

The files ``lab3.asm'' and ``lab6.asm'' tested the functionality of the label 
handling. Additionally, these programs as well as ``lab4.asm'' and ``lab5.asm'' 
included each instruction format, so the functionality of processing R-type, 
I-type, and J-type instructions was also tested. 

Although this assembler recognizes a larger MIPS ISA than just the instructions
that were tested, the implementation for each insuction format was the same. 
Then, testing each format using the programs provided was sufficient coverage 
for verifying functionality.

The files ``pseudo1.asm'' and ``pseudo2.asm'' tested functionality of 
pseudoinstructions. The first file tests all pseudoinstructions included in the
lookup table. The second file tests the integration of pseudoinstructions with
a program with other instructions and labels. 

% TODO: make this true
Expected output files are provided to check for differences with the output of
the assembler. For this assembler, all the tests used were passed.
% explain tests passed and proof of functionality

The use of lookup tables enabled the opcode, syntax group, and potential labels 
to each be found in constant time efficiency. 

\section*{Conclusion}
Improvements to this implementation of the assembler include adding more 
instructions and macro capabilities.

The set of instructions and pseudoinstructions can be easily expanded in this
implementation by adding to the lookup table. The infrastructure for 
implementing macros is also available from the struct that defines a portable
instruction. Then, a macro would be implemented like a pseudoinstruction in 
which the list of instructions to be run are set by the program instead of 
stored as constants in the lookup table.
\end{document}
